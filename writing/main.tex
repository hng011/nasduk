\documentclass[a4paper]{article}
\documentclass[a4paper]{article}
\usepackage[utf8]{inputenc}
\usepackage{times}
\usepackage{graphicx}
\usepackage{amsmath}
\usepackage{setspace}
\usepackage{array}
\usepackage{tocloft}
\usepackage[paperwidth=21.5cm, paperheight=29.7cm, top=4cm, bottom=3cm, left=4cm, right=3cm]{geometry}

\newcommand{\fontChapterTitle}{30pt}
\newcommand{\fontSubChapterTitle}{12pt}

\newcommand{\thesisTitle}{\uppercase{None}}
\newcommand{\name}{Hans Naufal Granito}
\newcommand{\npm}{50421591}
\newcommand{\signDate}{Jakarta, Day Month 2025}

\newcommand{\buLintang}{Prof. Dr. Lintang Yuniar Banowosari, SKom., MSc.}
\newcommand{\buElfitrin}{Dr. Elfitrin Syahrul, ST., MT.}
\newcommand{\buMarliza}{Dr. Marliza Ganefi Gumay, Skom., MMSI.}
\newcommand{\pakRavi}{Dr. Ravi Ahmad Salim}
\newcommand{\pakWahyudi}{Prof. Dr. Wahyudi Priyono}
\newcommand{\buMargianti}{Prof. Dr. E.S. Margianti, SE., MM.}
\newcommand{\pakAdang}{Prof. Dr.-Ing. Adang Suhendra SSi., SKom., MSc.}
\newcommand{\pakEdi}{Dr. Edi Sukirman, SSi., MM., M.I., Kom.}
\newcommand{\mySupervisor}{None}

\setlength{\fboxrule}{2pt}

\renewcommand{\baselinestretch}{1.5} % set line spacing

\begin{document}
    \fontsize{\fontSubChapterTitle}{0.5cm}\selectfont % set font to 12pt for all pages
    \pagestyle{empty}
    \pagenumbering{roman}
    \setcounter{page}{1} 

    % Title Page 
    \addcontentsline{toc}{section}{Halaman Judul}
    \begin{center}
        {\fontsize{\fontChapterTitle}{0.5cm}\selectfont 
        \textbf{UNIVERSITAS GUNADARMA}} \\
         
        \textbf{FAKULTAS TEKNOLOGI INDUSTRI} \\ [3em]
        
        \includegraphics[width=6cm, height=6cm, keepaspectratio]{assets/logo-gundar.png} \\ [1em]
        
        \text{SKRIPSI} \\[2em]
    
        % Details in a Box
        \fbox{
        \begin{minipage}{0.85\textwidth}  % Box width
            \newcommand{\vhspaceval}{0.3cm}
            \vspace{\vhspaceval}  % Vertical padding
            \begin{center}
                \thesisTitle
            \end{center}
            \hspace{\vhspaceval}
            \begin{tabular}{ll}
                \text{Nama} & : \name \\ 
                \text{NPM} & : \npm \\ 
                \text{Fakultas} & : Teknologi Industri \\ 
                \text{Jurusan} & : Informatika \\ 
                \text{Pembimbing} & : \mySupervisor \\ 
            \end{tabular}
            \vspace{\vhspaceval}
        \end{minipage}
        } \\[4em]

        \text{Diajukan Guna Melengkapi Sebagian Syarat}\\
        \text{Dalam Mencapai Gelar Sarjana Strata Satu (S1)}\\
        \textbf{JAKARTA}\\
        \textbf{2025}
        
    \end{center}

    % Originality & Publication statement 
    \newpage
    \pagestyle{plain}
    
    {\centering
        {\section*{
            \fontsize{\fontChapterTitle}{0}\selectfont
                Pernyataan Orisinalitas dan Publikasi
            } 
            \addcontentsline{toc}{section}{Pernyataan Orisinalitas dan Publikasi}
        }
    }
    
    Saya yang bertanda tangan di bawah ini :\\[1em]
    \begin{tabular}{p{3cm}p{0.1cm}p{10cm}}
        \text{Nama} & : & \name \\ 
        \text{NPM} & : & \npm \\
        \text{Judul Skripsi} & : & \thesisTitle \\ 
        \text{Tanggal Sidang} & : & None \\
        \text{Tanggal Lulus} & :  & None \\[2em]
    \end{tabular} 

    Menyatakan bahwa tulisan di atas merupakan hasil karya saya sendiri dan 
    dapat dipublikasikan sepenuhnya oleh Universitas Gunadarma. Segala kutipan 
    dalam bentuk apapun telah mengikuti kaidah dan etika yang berlaku. Semua hak 
    cipta dari logo serta produk yang disebut dalam buku ini adalah milik 
    masing-masing pemegang haknya, kecuali disebutkan lain. Mengenai isi dan 
    tulisan merupakan tanggung jawab Penulis, bukan Universitas Gunadarma.\\

    Demikianlah pernyataan ini dibuat dengan sebenar-benarnya dan 
    dengan penuh kesadaran \\[2em]

    \begin{flushleft}
         \begin{tabular}{c}
            \signDate \\ [4em]
            (\name) \\
         \end{tabular} 
    \end{flushleft}

    % Approval
    \newpage
     
    \begin{center}

        {\section*{
            \fontsize{\fontChapterTitle}{0}\selectfont
                Lembar Pengesahan \\[2em]
            } 
            \addcontentsline{toc}{section}{Lembar Pengesahan}
        }
        
        \newcommand{\spaceTitle}{0.5em}
        \textbf{KOMISI PEMBIMBING} \\[\spaceTitle]    
        \begin{tabular}{|c|p{10cm}|c|}
            \hline 
                \multicolumn{1}{|c|}{\textbf{NO}} & 
                \multicolumn{1}{c|}{\textbf{NAMA}} & 
                \multicolumn{1}{c|}{\textbf{KEDUDUKAN}} \\
            \hline 
                1. & \buLintang & Ketua \\
            \hline 
                2. & \buElfitrin & Anggota \\
            \hline 
                3. & \buMarliza  & Ketua \\
            \hline
        \end{tabular} \\[3em]

        \textbf{PANITIA UJIAN} \\[\spaceTitle]
        \begin{tabular}{|c|p{10cm}|c|}
            \hline 
                \multicolumn{1}{|c|}{\textbf{NO}} & 
                \multicolumn{1}{c|}{\textbf{NAMA}} & 
                \multicolumn{1}{c|}{\textbf{KEDUDUKAN}} \\
            \hline 
                1. & \buLintang & Ketua \\
            \hline 
                2. & \buElfitrin & Sekretaris \\
            \hline 
                3. & \buLintang  & Anggota \\
            \hline 
                4. & \buElfitrin  & Anggota \\
            \hline 
                5. & \buMarliza  & Anggota \\
            \hline
        \end{tabular} \\[4em]

        \text{Mengetahui,} \\[1em]
    
    \end{center} 

    \noindent {
        \parbox{0.4\linewidth}{
            \centering Pembimbing \\[2cm]
            (\mySupervisor)
        }
        \parbox{0.6\linewidth}{
            \centering Bagian Sidang Ujian \\[2cm] % Add space for the signature
            (\pakEdi)
        }
    }

    % Abstract
    \newpage

    {\centering
        {\section*{
            \fontsize{\fontChapterTitle}{0}\selectfont
                Abstrak \\[1em]
            } 
            \addcontentsline{toc}{section}{Abstrak}
        }
    }

    % Acknowledgments 
    \newpage

    {\centering
        {\section*{
            \fontsize{\fontChapterTitle}{0}\selectfont
                Kata Pengantar \\[1em]
            } 
            \addcontentsline{toc}{section}{Kata Pengantar}
        }
    }

    Segala puji dan syukur penulis panjatkan ke hadirat Allah SWT. Yang Maha Kuasa,
    atas segala berkat, anugerah, dan karunia-Nya yang melimpah. Penulis bersyukur 
    dapat menyelesaikan Tugas Akhir ini dengan penuh rasa syukur dan terima kasih kepada
    Allah SWT.

    Tugas akhir ini disusun sebagai salah satu syarat untuk memperoleh gelar Sarjana
    Komputer dari Universitas Gunadarma. Judul Tugas Akhir ini adalah "\thesisTitle" 

    Dalam perjalanan menyusun Tugas Akhir ini, penulis menyadari bahwa banyak rintangan
    dan kesulitan yang harus dihadapi. Namun berkat bantuan dan dukungan, serta dorongan
    dari berbagai pihak, Tugas Akhir ini dapat diselesaikan dengan baik. Penulis ingin
    mengucapkan terima kasih yang tak terhingga kepada:

    \begin{enumerate}
        \item \buMargianti, selaku Rektor Universitas Gunadarma.
        \item \pakAdang, selaku Dekan Fakultas Teknologi Industri Universitas Gunadarma
        \item \buLintang, selaku Ketua Program Studi Informatika Universitas Gunadarma
        \item \pakEdi, selaku Kepala Bagian Sidang Ujian Universitas Gunadarma
        \item \mySupervisor, selaku Dosen Pembimbing yang telah banyak memberikan 
        bimbingan, arahan, waktu, tenaga, dan kebaikan hati untuk membantu penulis dalam
        menyelesaikan penulisan ini
        \item Kedua orang tua beserta adik yang senantiasa mendukung penulis baik secara mental maupun materil
        dalam menyelesaikan penulisan ini.
        \item Teman-teman 4IA12 yang banyak memberikan dukungan berupa kritik dan saran 
        sehingga penulisan ini dapat diselesaikan dengan baik. 
    \end{enumerate}

    % Table of Contents
    \newpage

    {\centering
        {\section*{
            \fontsize{\fontChapterTitle}{0}\selectfont
                Daftar Isi \\[1em]
            } 
            \addcontentsline{toc}{section}{Daftar Isi}
        }
        
        \renewcommand{\cftdotsep}{1} % Adds dots
        \renewcommand{\cftsecleader}{\cftdotfill{\cftdotsep}} % Adds dots
        \renewcommand{\contentsname}{}
        \setlength{\cftbeforesecskip}{3pt} % Adjust spacing between section entries
        
        \tableofcontents

    }

    % List of Tables
    \newpage

    {\centering
        {\section*{
            \fontsize{\fontChapterTitle}{0}\selectfont
                Daftar Tabel \\[1em]
            } 
            \addcontentsline{toc}{section}{Daftar Tabel}
        }
    }

    % List of Figures
    \newpage
    {\centering
        {\section*{
            \fontsize{\fontChapterTitle}{0}\selectfont
                Daftar Gambar \\[1em]
            } 
            \addcontentsline{toc}{section}{Daftar Gambar}
        }
    }

    % List of Appendicies
    \newpage
    {\centering
        {\section*{
            \fontsize{\fontChapterTitle}{0}\selectfont
                Daftar Lampiran \\[1em]
            } 
            \addcontentsline{toc}{section}{Daftar Lampiran}
        }
    }

    % Bagian Isi
    % Bab 1 Pendahuluan
     % Bagian Isi
    % Bab 1 Pendahuluan
    \newpage

    % reset the chapter numbering
    
    \pagenumbering{arabic}

    {\centering
        \renewcommand{\thesection}{\arabic{section}.}
        \section{Pendahuluan}
    }

    % Latar Belakang
    \subsection{Latar Belakang Masalah}

    % Ruang Lingkup
    \subsection{Ruang Lingkup}

    % Tujuan Penelitian
    \subsection{Tujuan Penelitian}

    % Sistematika Penulisan
    \subsection{Sistematika Penulisan}



    % Bab 2 Tinjauan Pustaka
        % Bab 2 Tinjauan Pustaka
    \newpage

    {\centering
        \renewcommand{\thesection}{\arabic{section}.}
        \section{Tinjauan Pustaka}
    }

    \subsection{TODO}
    test 321 ASAS TODO \textbf{TODO}

    \subsubsection{TODO}
    test 123 ASAS TODO \textbf{TODO}

    \subsection{TODO}
    \subsection{TODO}



    % Bab 3 Metode Penelitian
        % Bab 3 Metode Penelitian
    \newpage
    
    \setcounter{section}{3}
    \setcounter{subsection}{0}
    {\centering
        \renewcommand{\thesection}{\arabic{section}. }
        \section*{
            \fontsize{\fontChapterTitle}{0}\selectfont
            \thesection Metode Penelitian
        }
        \addcontentsline{toc}{section}{3. Metode Penelitian}
    }

    % Bab 4 Hasil dan Pembahasan
        % Bab 4 Hasil dan Pembahasan
    \newpage

    {\centering
        \renewcommand{\thesection}{\arabic{section}.}
        \section{
            \fontsize{\fontChapterTitle}{0}\selectfont
            Hasil dan {Pembahasan}
        }
    }



    % Bab 5 Penutup
        % Bab 5 Penutup
    \newpage

    {\centering
        \renewcommand{\thesection}{\arabic{section}.}
        \section{Penutup}
    }



    % Bagian Akhir
    % Daftar Pustaka
    \newpage

    {\centering
        \section*{
            \fontsize{\fontChapterTitle}{0}\selectfont
            Daftar Pustaka\\[1em]
        }
        \addcontentsline{toc}{section}{Daftar Pustaka}
    }

    % Lampiran
    \newpage

    {\centering
        \section*{
            \fontsize{\fontChapterTitle}{0}\selectfont
            Lampiran\\[1em]
        }
        \addcontentsline{toc}{section}{Lampiran}
    }

\end{document}
